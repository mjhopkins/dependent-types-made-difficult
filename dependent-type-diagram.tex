\documentclass[11pt]{beamer}
\usepackage{geometry}                % See geometry.pdf to learn the layout options. There are lots.
\geometry{letterpaper}                   % ... or a4paper or a5paper or ... 
%\geometry{landscape}                % Activate for for rotated page geometry
%\usepackage[parfill]{parskip}    % Activate to begin paragraphs with an empty line rather than an indent
\usepackage{graphicx}
\usepackage{amssymb}
\usepackage{epstopdf}
\usepackage{amssymb}
\usepackage{amsfonts}

\usepackage{tikz}
\usetheme{metropolis}

\DeclareMathOperator{\Set}{Set}
\DeclareMathOperator{\Hom}{Hom}
\DeclareMathOperator{\Vect}{Vect}
\DeclareMathOperator{\Type}{Type}
\DeclareMathOperator{\Nat}{Nat}
\DeclareMathOperator{\Ob}{Ob}
\DeclareMathOperator{\Lan}{Lan}
\DeclareMathOperator{\Ran}{Ran}
\DeclareMathOperator{\ANZ}{ANZ}
\DeclareMathOperator{\NAB}{NAB}
\DeclareMathOperator{\CBA}{CBA}
\DeclareMathOperator{\Westpac}{Westpac}
\DeclareMathOperator{\Bank}{Bank}
\DeclareMathOperator{\IsEven}{IsEven}

\begin{document}

\begin{frame}
  \frametitle{Dependent types}
  \begin{align*}
    \IsEven&: \mathbb{N} \to \Set  \\
    \IsEven &\, 0 = \{ \textrm{zeroEven} \} \\
    \IsEven &\, 1 = \{ \textrm{} \} \\
    \IsEven &\, 2 = \{ \textrm{twoEven} \} \\
    \IsEven &\, 3 = \{ \textrm{} \} \\
    \IsEven &\, 4 = \{ \textrm{fourEven} \} \\
    &\vdots
  \end{align*}
  \begin{align*}
    \textrm{CustomerOfBranch}&: \Bank \to \Set  \\
    \textrm{CustomerOfBranch}& \ANZ = \{ \textrm{Alice, Albert, Anne-Marie} \} \\
    \textrm{CustomerOfBranch}& \CBA = \{ \textrm{Calvin, Colin} \} \\
    \textrm{CustomerOfBranch}& \NAB = \{ \textrm{Nathan} \} \\
    \textrm{CustomerOfBranch}& \Westpac = \{ \textrm{Wendy, Will} \} \\
  \end{align*}

\end{frame}

\begin{frame}

  \begin{tikzpicture}
  \def\x{2}
    \begin{scope}[yshift=6.5cm]
    \draw [fill=yellow] (0,0.5) rectangle (4, 3.5);
    \node[right] at (\x+2, 3) {ANZ};
    \node at (\x, 3) {Alice};
    \node at (\x, 2) {Albert};
    \node at (\x, 1) {Anne-Marie};
    \end{scope}
    \begin{scope}[yshift=4cm]
    \draw [fill=yellow] (0,0.5) rectangle (4, 2.5);
    \node[right] at (\x+2, 2) {CBA};
    \node at (\x, 2) {Calvin};
    \node at (\x, 1) {Colin};
    \end{scope}
    \begin{scope}[yshift=2.5cm]
    \draw [fill=yellow] (0,0.5) rectangle (4, 1.5);
    \node[right] at (\x+2, 1) {NAB};
    \node at (\x, 1) {Nathan};
    \end{scope}
    \begin{scope}[yshift=0cm]
    \draw [fill=yellow] (0,0.5) rectangle (4, 2.5);
    \node[right] at (\x+2, 2) {Westpac};
    \node at (\x, 2) {Wendy};
    \node at (\x, 1) {Will};
    \end{scope}
  \end{tikzpicture}
  %
  \begin{tikzpicture}
  \def\x{2}
    \begin{scope}[yshift=7.5cm]
      % \draw [fill=yellow] (0,0.5) rectangle (4, 1.5);
      % \node at (\x+2, 1) {4};
      \node at (\x, 1) {\vdots};
    \end{scope}  
    \begin{scope}[yshift=6cm]
      \draw [fill=yellow] (0,0.5) rectangle (4, 1.5);
      \node[right] at (\x+2, 1) {4};
      \node at (\x, 1) {fourEven};
    \end{scope}  
    \begin{scope}[yshift=4.5cm]
      \draw [fill=yellow] (0,0.5) rectangle (4, 1.5);
      \node[right] at (\x+2, 1) {3};
    \end{scope}  
    \begin{scope}[yshift=3cm]
      \draw [fill=yellow] (0,0.5) rectangle (4, 1.5);
      \node[right] at (\x+2, 1) {2};
      \node at (\x, 1) {twoEven};
    \end{scope}  
    \begin{scope}[yshift=1.5cm]
      \draw [fill=yellow] (0,0.5) rectangle (4, 1.5);
      \node[right] at (\x+2, 1) {1};
    \end{scope}  
    \begin{scope}[yshift=0]
      \draw [fill=yellow] (0,0.5) rectangle (4, 1.5);
      \node[right] at (\x+2, 1) {0};
      \node at (\x, 1) {zeroEven};
    \end{scope}  
    % \node at (\x, 2) {twoEven};
    % \node at (\x, 0) {zeroEven};
  \end{tikzpicture}


\end{frame}

\begin{frame}
vvv


  \begin{tikzpicture}[scale=2.7]
    \draw (0,0) -- (90:1);
    \draw (0,0) -- (210:1);
    \draw (0,0) -- (330:1);
    \def\r{4}
    %90 210 330
    %150 270 390=30
    \draw (150:.\r) node {Logic};
    \draw (30:.\r)  node {Types};
    {{\draw (270:.\r) node {Categories};}}
    \end{tikzpicture}
\end{frame}


\end{document}  